\documentclass[a4paper,12pt]{report}

% https://tex.stackexchange.com/a/63393
\makeatletter
\def\@makechapterhead#1{%
  %%%%\vspace*{50\p@}% %%% removed!
  {\parindent \z@ \raggedright \normalfont
    \ifnum \c@secnumdepth >\m@ne
        \huge\bfseries \@chapapp\space \thechapter
        \par\nobreak
        \vskip 20\p@
    \fi
    \interlinepenalty\@M
    \Huge \bfseries #1\par\nobreak
    \vskip 40\p@
  }}
\def\@makeschapterhead#1{%
  %%%%%\vspace*{50\p@}% %%% removed!
  {\parindent \z@ \raggedright
    \normalfont
    \interlinepenalty\@M
    \Huge \bfseries  #1\par\nobreak
    \vskip 40\p@
  }}
\makeatother

% spacings
\usepackage[a4paper,right=15mm,left=25mm,top=20mm,bottom=20mm]{geometry}

% 1.5 line spacing
\usepackage{setspace}
\setstretch{1.5}

% Being able to copy-paste from the document
% Display the diacritics correctly
\usepackage{fontspec}
% Times New Roman font
\setmainfont{Times New Roman}

% Romanian language support
\usepackage[romanian]{babel}

% https://tex.stackexchange.com/a/57967 
% \input{glyphtounicode}
% \pdfgentounicode=1

% \usepackage{mathptmx}

% Appendices
\usepackage[toc,page]{appendix}

% Lista abrevierelor
\usepackage[printonlyused]{acronym}

% Format urls in the bibliography
% \usepackage{url}
\usepackage{hyperref}
% Clickable table of contents
\hypersetup{linktoc=all}

% Code blocks
\usepackage{minted}
\usepackage{color}
\setminted{%
  autogobble=true,
  codetagify=true,
  linenos=true,
  breaklines=true,
  baselinestretch=0.8,
  % https://www.giss.nasa.gov/tools/latex/ltx-178.html
  fontsize=\footnotesize
  % Prevent it from spilling over margins for especially long lines
  ,breakanywhere=true
}

% Chapters should count with roman numerals
\renewcommand{\thechapter}{\Roman{chapter}}
\renewcommand{\thesection}{\arabic{chapter}.\arabic{section}}
\renewcommand{\thesubsection}{\arabic{chapter}.\arabic{section}.\arabic{subsection}}

% Add dots to numbers, only applies to TOC
\let\savenumberline=\numberline
\def\numberline#1{\savenumberline{#1.}}

% Chapters must left aligned and 14 pt.
% There's no 14 pt available, though, only 14.4 (\large)
% https://tex.stackexchange.com/questions/24599/what-point-pt-font-size-are-large-etc/24600#24600
\usepackage{titlesec}    
% \titleformat{〈command 〉}[〈shape〉]{〈format〉}{〈label 〉}{〈sep〉}{〈before-code〉}[〈after-code〉]
% https://mirror.marwan.ma/ctan/macros/latex/contrib/titlesec/titlesec.pdf#page=4&zoom=200,67,627
% It gets pretty involved...
\def\tabSize{1cm}
\titleformat{\chapter}[block]{\normalfont\large\bfseries\filcenter}{\thechapter~}{\tabSize}{}
\titleformat{\section}[block]{\normalfont\normalsize\bfseries\filright}{\thesection~}{\tabSize}{}
\titleformat{\subsection}[block]{\normalfont\normalsize\bfseries\filright}{\thesubsection~}{\tabSize}{}

% Skips before and after the title
\titlespacing*{\chapter}{0pt}{12pt}{6pt}
\titlespacing*{\section}{0pt}{12pt}{6pt}
\titlespacing*{\subsection}{0pt}{12pt}{6pt}

% Aliniat de 1 cm.
\usepackage{indentfirst}
\setlength{\parindent}{1.0cm}

% https://tex.stackexchange.com/questions/163451/total-number-of-citations
\usepackage{totcount}
\newtotcounter{citnum} %From the package documentation
\def\oldbibitem{} \let\oldbibitem=\bibitem{}
\def\bibitem{\stepcounter{citnum}\oldbibitem}

% Thank you to @paante on LaTeX Discord.
% Makes it possible to mark pages and then do math on them.
% usage: \markpage{key}
%        \getpagemark{key}
\newwrite\pagecountf
\AtBeginDocument{%
    \InputIfFileExists{\jobname.pagecount}{}{}%
    \immediate\openout\pagecountf={\jobname.pagecount}%
}
\AtEndDocument{\closeout\pagecountf}
\protected\def\markpage#1{%
    \write\pagecountf{%
        \noexpand\expandafter\gdef
            \noexpand\csname pagecount/#1\endcsname
            {\the\value{page}}%
    }%
}
\def\getpagemark#1{\ifcsname pagecount/#1\endcsname \csname pagecount/#1\endcsname \else 0\fi}


% Extracting the name of chapters.
% https://tex.stackexchange.com/questions/62241/how-to-get-the-current-chapter-name-section-name-subsection-name-etc
\usepackage{nameref}

% No spacing for lists.
% https://tex.stackexchange.com/questions/10684/vertical-space-in-lists
\usepackage{enumitem}
\setlist{nosep}

\newcommand{\unnumberedChapter}[1]{%
  \chapter*{#1}
  \addcontentsline{toc}{chapter}{#1}}

% Description
\newcommand{\authorName}{Curmanschii Anton}
\newcommand{\documentTitle}{Raport la practică de producție}
\newcommand{\uniGroupName}{IA1901}
\newcommand{\programulDeStudii}{licență}

% Start
\begin{document}

% Prevent spills over margin
\sloppy

\begin{titlepage}
  \vspace*{\fill}
  \begin{center}
      \vspace*{1cm}

      \large
      \uppercase{\textbf{UNIVERSITATEA DE STAT DIN MOLDOVA\\}}

      \normalsize
      \uppercase{\textbf{FACULTATEA DE MATEMATICĂ SI INFORMATICĂ}}
      \vspace{0.1cm}

      \normalsize
      \uppercase{\textbf{SPECIALITATEA INFORMATICA APLICATĂ}}
      \vspace{3.0cm}

      \large
      \textbf{\uppercase\expandafter{\authorName}}
      \vspace{1.5cm}

      \Large
      \textbf{\uppercase\expandafter{\thesisTitle}}
      \vspace{0.75cm}

    \end{center}
  \vfill

  \normalsize

  \begin{flushleft}
    \begin{tabular}{ p{4cm} p{4cm} p{8cm} }
      Șef de departament:      & \underscores{4cm} & IDK Who \\
      Conducătorul științific: & \underscores{4cm} & MARIN Maria \\
      Autor:                   & \underscores{4cm} & \authorName
    \end{tabular}
  \end{flushleft}

  \begin{center}
    \textbf{Chișinău -- 2022}
  \end{center}

\end{titlepage}

\clearpage
\tableofcontents

\clearpage
\unnumberedChapter{Lista abrevierelor}
\begin{acronym}
  \begin{center}
    \acro{XR}{Extended Reality}
    \acro{VR}{Virtual Reality}
    \acro{3D}{3-Dimensional}
    \acro{XML}{Extended Markup Language}
    \acro{UI}{User Interface}
  \end{center}
\end{acronym}


\chapter{Detaliile privind locul de practică}


\section{Laboratorul AVR}

Practica s-a petrecut la universitate, la laboratorul AVR.
Laboratorul acesta este o sală de calculatori, destul de largă, conținând mai mult ca 15 de calculatori moderne și puternice, un printer \ac{3D}, echipamentul \ac{XR} divers, adică căștile și controller-uri \ac{VR}.
Această sală este perfectă pentru elaborarea eficientă și avansată a jocurilor video.

Sala de calculători era deschisă în aproape orice zi de lucru, de la 11:00 dimineața până 20:00 seara, și am avut posibilitate s-o folosesc spre folosul meu.


\section{Procesul de lucru la laborator}

Am primit o sarcină concretă --- să se elaboreze un joc cu un set de cerințe fixat, reușind să realizez proiectul în timpul prestabilit.
La dispoziție am avut orice echipament necesar din laborator, și am putut să pun întrebări la conducător de practică.

Deoarece jocul s-a planificat să fie un joc simplu pentru calculator, nu am folosit alt echipament decât propriu zis calculător.
Datorită faptului că calculatorii sunt atât de bune, am putut să lucrez foarte eficient asupra sarcinii.

Orarul a fost liber --- am putut să vin sau să nu vin în orice zi, principalul a fost să reușesc să prezint proiectul în termen limită.
Când puteam, veneam la universitate la ora 11:00 și lucram până la 19:00-20:00 seara.
În zilele de odihnă și în acele zile când nu am fost la Chișinău, lucram din casă, de pe laptopul propriu.

Planificarea mai detaliată a realizării sarcinii am administrat în cea mai mare parte eu singur.
Eu mă descurcam cu implementarea sarcinii singur, de aceea aici nu a fost necesar ajutorul din partea conducătorului.


\chapter{Sarcina de practică}

După cum s-a menționat anterior, s-a cerut să se elaboreze un joc cu un set de cerințe fixat.
Jocul concret care urmat să fie implementat a fost un joc de curse cu mașini, cu o vedere la persoana a treia.
Jocul trebuia să aibă 2 scene, cu cerințe specifice la fiecare scenă:

\textbf{Garaj:}
\begin{itemize}
  \item Posibilitatea de a schimba culoarea mașinii;
  \item Un sistem de caracteristici (stat-uri) cu sănătatea și viteza;
  \item Posibilitatea de a selecta o mașină dintr-o listă prestabilită;
  \item Posibilitatea de a-și selecta numele de utilizator;
  \item Serializarea selectării culorii, mașinii, numii și a stat-urilor, folosind \ac{XML} și \texttt{PlayerPrefs} în Unity;
  \item Posibilitatea de a adjusta stat-urile utilizând valuta;
  \item Posibilitatea de a adăuga valuta prin tranzacții cu bani reali (simulat, fără tranzacții reale);
  \item Posibilitatea de a selecta una din două mape și a începe al doilea stagiu al jocului.
\end{itemize}

\textbf{Tranziția între scene:}
\begin{itemize}
  \item Să fie arătat un avertisment în timpul tranziției.
\end{itemize}

\textbf{Gameplay:}
\begin{itemize}
  \item Să fie realizat un joc de curse cu două mașini, una controlată de jucător, alta de computer;
  \item Să se utilizeze \texttt{Addressables} pentru a spawna mașinile.
  \item Mașina poate să se accelereze, să se frâneze, să se vireze la dreapta și la stânga;
  \item Să se realizeze un controller pe baza vitezelor, ca în mașini reale. Să se permită a schimba viteze între 1 și 5, și să fie și o viteză inversă;
  \item Caracteristicile specifice a motorului să se depindă de stat-uri selectate în garaj;
  \item Prima mașină care ajunge la sfârșitul traseului câștigă și cursul se termină, după ce să se întoarcă la garaj.
\end{itemize}


\chapter{Realizarea proiectului de practică}


\section{Analiza mai profundă a cerințelor}

Chiar la începutul lucrului, mi-a fost clar că proiectul este destul de voluminos, dar în același timp ușor de abordat, neavând ceva extraordinar complicat de realizat.
După ce am studiat și am reflectat despre cerințele proiectului, am ajuns la următoarele concluzii care într-un fel au marcat și planul lucrării ce să se urmeze.

Garaj trebuia să fie simplu de implementat: trebuia să creez sau să găsesc unele modele de mașini, să le includ în joc, să creez elementele \ac{UI}-ului, să le conectez la modelul cu date prin event-uri.
Aici nu-i nimic extraordinar, doar trebuie să se implementeze logica și să se conecteze \ac{UI}-ul la event-uri.

Cu scena principală cu gameplay eu nu am fost sigur cum s-o realizez, deoarece nu am știut cum anume lucrează sistemul de viteze în mașini, nu am cunoscut sistemul Addressables și nu am știut cum să realizez mecanicile mașinilor în Unity.
Am avut noroc cu ultimul punct, deoarece Unity are collider-uri incluse pentru roți care realizează majoritatea calculărilor fizice complicate necesare.

Ca să fiu mai productiv, am decis să conectez generatorul meu de cod la proiect, să integrez aproape imediat consola integrată care am programat-o anterior, și să-mi setez un mediu pentru crearea posibilă a instrumentelor, aparte de proiect în Unity.
Anterior deja am programat un astfel de mediu, sau un astfel de instrument de productivitate.
A fost programat în Python, și, cu toate că lucra destul de bine, a fost destul de dificil de menținut.
De aceea, am hotărât să dedic vreo săptămână la programarea acestui mediu și setarea infrastructurii generale a proiectului.


\section{Proiectarea infrastructurii proiectului}

În primul rând, am hotărât să încep cu instrumentul \texttt{dev} de productivitate, în care să integrez totul legat de adminisitrarea proiectului: compilarea generatorului de cod, rularea generatorului de cod, setarea inițială a proiectului, și posibil mai multe lucruri ca build sau testarea automatizate, publicarea pachetelor automatizate, script-uri de automatizare pentru convertarea între formate, script-uri de verificare a validității stării proiectului, etc.

Am hotărât să realizez instrumentul dat în limbajul D, deoarece consider că este o alegere bună pentru crearea script-urilor: poate fi de nivel foarte înalt, este foarte expresiv și are un \ac{GC} util pentru script-uri, dar în același timp are un sistem static de tipuri și este compilat.
Pentru a realiza un astfel de instrument, însă, mi-a trebuit și o librărie capabilă pentru instrumente pentru linie de comandă (CLI Framework).
Am găsit framework-ul lui Bradley Chatha, JCLI, însă am hotărât să lucrez asupra ei, deoarece a lipsit suportul pentru comenzi imbricate, deciziile în cod nu a fost documentate destul de bine, și sintaxa de utilizare a fost incomodă.
Aceasta a terminat cu petrecerea a aproximativ 5 zile rescriind modulele de bază, simplificându-le și făcând părțile aceasta mai reutilizabile în general, și adăugând pe urma capacitățile necesare mie.

După aceasta, am creat instrumentul \texttt{dev} într-un timp scurt, integrând și generatorul de cod cu consola.

Astfel, etapa inițială de creare a mediului de dezvoltare s-a terminat.


\section{Garaj}

Am început lucrul asupra scenei cu realizarea că voi avea nevoie de câteva modele de mașini.
Am încărcat niște modele, dar am realizat că ele au și configurații foarte diferite.
Că să fie mai ușor de lucrat cu ele din cod, a fost necesar să le preprocesez manual:

\begin{itemize}
  \item Să renumesc roțile conform poziției lor (front-right, front-left, back-right, back-left);
  \item Să pun toate roțile sub un singur părinte comun;
  \item Să renumesc obiectul cu corpul mașinii în \texttt{body}
  \item Dacă modelul este prea detaliat, să-l simplific;
  \item Să le rotesc, punandu-le într-o orientație comună.
\end{itemize}

Am decis dar să studiez baza aplicației de modelare Blender pentru a putea face aceste modificări, creând un model de mașină al meu propriu.
Nu a fost dificil pentru mine, deoarece deja cunosc baza lui \ac{3D} rendering, cum se reprezintă și se rasterează obiectele în \ac{GPU}, mi-a rămas deci doar să învăț cum să folosesc instrumente pentru modelare.
Cu încărcarea modelelor am avut puține probleme.

După aceasta, am creat modelul de date a mașinii și a utilizatorului, am implementat serializarea în \ac{XML} și în PlayerPrefs respectiv, după ce am implementat celelalte sisteme și le-am conectat la modelul de date prin event-uri.

Una din cerințe funcționale a fost de a putea reseta culoarea mașinii.
Pentru aceasta am avut nevoie de un color picker --- elementul \ac{UI} pentru selectarea culorii.
Am găsit codul care să implementeze acesta, însă, realizând cât de prost, complicat și neeficient este codul dat, l-am rescris din nou, simplificându-l, dar păstrând toate funcționalități le.


\section{Gameplay}

În acel moment, proiectul a ajuns la o etapă unde a trebuit deja să încep a căuta informații despre modul în care pot implementa fizica mașinilor în Unity.
După cum s-a menționat, am avut noroc cu \texttt{WheelCollider} care există în Unity implicit.
Am realizat doar un script care să le seteze corect pe baza modelelor mașinilor, descoperind roții, mărimea lor, și setând collider-uri conform acestor date.

Cu mecanicile motorului și ale vitezelor am avut mai puțin noroc.
Aici, trebuiam să fac research singur, de la zero.
După studierea informațiilor de pe internet am înțeles cum lucrează vitezele și eficiența motorului și le-am putut implementa în cod.
Cu aceste lucruri, și mai ceva setup, mi-am putut testa codul în Unity.

Am decis să realizez un fel de speedometer și tachometer și încă un fel de display pentru viteză curentă.
Am creat o componentă grafică care desenează un cerc cu diviziuni, generată dinamic pe baza configurației care include lățimea și lungimea lor, precum și cât de dese sunt.
Creez câte o astfel de componentă pentru viteza și pentru revoluțiile motorului per minut care sunt controlate de scripturi respective, care le leagă de datele curente despre mașină.

Următorul pas a fost implementarea logicii cursului și a mapelor.
După jucarea cu sistemul terrain-ului în Unity, și cu pachetele pentru drumuri, am înțeles că aș avea nevoie de mai mult timp ca să realizez niște mape interesante și să creez interfețe pentru logica pachetelor la codul meu, de aceea am hotărât să fac două mape unde drumul se duce doar înainte, și nu avem decorații.
Doar schimb culoarea background-ului între mape.

Am programat după aceasta sistemul cursului, unde dacă jucătorul trece de pe drum, el este respawnat cu viteza zero, și câștigă când sfârșește cursul.
Depind de o interfață abstractă care dă informații pentru drumul pe baza segmentelor.
Fiecare segment trebuie să fie asociat un ID, iar poziția jucătorului în segment poate fi expresată prin progresia în scopul acestuia.
Track-ul expune o interfață care permite conversiunea între sistemul de coordonate Unity și sistemul intern de segmente de drumuri, și pemite accesarea informațiilor despre segmente specifice a drumului, ca direcția normală, direcția drept, și accesarea despre întregul track, de exemplu, progresul total pe baza segmentului curent, oare mașina a trecut de pe drup.
Pentru mapa mea dreaptă, implementarea interfeței a fost foarte simplă --- am un singur segment drept de drum, dar se poate imagina un sistem de drumuri băzate pe curbe Bézier care tot ar integra bine cu sistemul acesta.


\section{Inițializarea și Tranzițiile}

Deoarece ambele scene deja au fost în mare parte realizate, am putut să încep a lurca asupra sistemului de tranziții între scene și asupra sistemului de inițializare.
Am decis să fac câte un script pentru inițializarea locală a scenei, adică acela care este folosit în varianta scenei unde ea ar exista în izolație,
și câte un script de inițializare normală, adică cea folosită în propriu zis joc.
Acest script va fi folosit de către scriptul central de inițializare a întregului joc.

Sună destul de simplu, însă inițializarea este un lucru care este destul de greu de implementat,
în special astfel încât să fie posibil să refolosească părți individuale ale inițializării,
și să se poată testa lucruri individuale într-un context izolat de întregul sistem.
Eu am decis să nu folosesc framework-uri care să simplifice și automatizeze acest proces de inițializare (dependency injection framework),
dar să fac eu singur un astfel de framework într-un scop limitat.
Cazurile care nu se acoperă de acest mini-framework sunt procesate manual, într-un script aparte.

Scripturile de inițializare sunt după design izolate pentru fiecare scenă, și nu se cunosc unul pe altul.
Așadar, a fost necesar încă cel puțin un script care să transmită datele necesare pentru inițializare dintr-o scenă în altă.
De exemplu, în cazul în care se inițializează scena Gameplay din scena Garaj, datele care trebuie să fie transmise sunt datele despre mașina selectată, caracteristicile mașinii, ca puterea motorului, numărul de boți, dificultatea boților, mapa selectată, etc.
Acest script de tranziție deja va lua aceste date cum ele au fost sugerate de Garaj, le va transforma într-un format care Gameplay poate folosi pentru inițializare, și va inițializa scena Gameplay, transmitându-i aceste date.

Mi-a luat destul de mult timp să înțeleg cum trebuie să lucreze sistemul acesta, și implementarea lui a ieșit destul de intricată, și am lăsat-o într-o stare imperfectă, însă funcțională.


\unnumberedChapter{Concluzii și Recomandări}

În timpul implementării sarcinii practice am realizat că sunt capabil la dezvoltarea jocurilor,
având toate cunoștințele necesare pentru dezvoltarea eficientă în Unity, putând produce un sistem ușor de menținut și modificat.
Am mai realizat că Unity lipsește mai multe instrumente de confort pentru dezvoltator care trebuie să fie programate manual.
Deci, pentru un workflow de dezvoltare maxim eficient, ar trebui să dezvolt niște librării și extensiuni pentru editor, inclusiv și mai multe plugin-uri utile pentru generatorul meu de cod.

Atmosfera în laborator a fost foarte prietenoasă, iar calculatorii au fost o adiție minunată pentru productivitate.
Fără un calculator puternic aș pierde foarte mult timp simplu așteptând ca Unity să-mi recompileze script-urile. 

Nu am învățat atât de multe lucruri noi referitor la programare, în mare parte pur și simplu mi-am solidificat cunoștințele.



% Bibliography
\bibliographystyle{plain}
\bibliography{bibliography}

% Appendices
\appendix

% Number with arabic numbers instead of Roman
\renewcommand{\thechapter}{\arabic{chapter}}
% Prepend Anexa to section names, center them
\titleformat{\section}[block]{\normalfont\normalsize\bfseries\filcenter}{Anexa \thesection.~}{0pt}{}
% Every section on new page
% \newcommand{\sectionbreak}{\clearpage}

\end{document}